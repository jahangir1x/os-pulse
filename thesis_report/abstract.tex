% Abstract
\chapter*{Abstract}
\addcontentsline{toc}{chapter}{Abstract}

\onehalfspacing

OS-Pulse is a comprehensive, full-stack real-time system monitoring platform designed to address the limitations of traditional monitoring tools in the domains of security research, malware analysis, and behavioral analysis. Traditional system monitoring solutions lack real-time capabilities, comprehensive data visualization, and the ability to correlate multiple data sources in a unified dashboard, creating significant gaps in security analysis workflows.

This project implements a modern web-based monitoring platform that provides deep insights into system behavior through dynamic instrumentation and network monitoring capabilities. The system architecture follows a microservices pattern with three distinct tiers: a React-based presentation tier for data visualization, a Go-based application tier with PostgreSQL database for backend services, and a multi-agent data collection tier utilizing Python and TypeScript for specialized monitoring tasks.

The core innovation of OS-Pulse lies in its multi-agent architecture that employs Frida-based dynamic instrumentation for non-intrusive, real-time monitoring without requiring process restarts. The platform captures and correlates file operations, network traffic, and process creation events through strategic Windows API hooking, providing security researchers with a unified view of system behavior. Key features include real-time file system operation monitoring with content preview, comprehensive process management with parent-child relationship mapping, and network activity analysis with HTTP/HTTPS traffic interception.

The system implements a complete user workflow from sample upload to analysis, featuring an integrated noVNC viewer for virtual machine interaction, real-time event visualization with sub-second display latency, and flexible data export capabilities in multiple formats. The architecture ensures session isolation, secure file handling, and maintains comprehensive audit trails for security compliance.

Technical implementations include a repository-service-handler layered architecture in the backend, JSONB-based flexible event storage in PostgreSQL, and a modular React component architecture with TypeScript for type safety. Performance optimizations such as database indexing, connection pooling, and selective monitoring rules ensure the platform scales effectively for enterprise deployment scenarios.

The project demonstrates successful integration of modern web technologies with system-level monitoring capabilities, providing a foundation for future enhancements including cross-platform support, machine learning-based anomaly detection, and distributed monitoring capabilities. OS-Pulse represents a significant contribution to the field of system monitoring by bridging the gap between traditional desktop-based monitoring tools and modern web-based analysis platforms.

\textbf{Keywords:} System Monitoring, Dynamic Instrumentation, Malware Analysis, Real-time Monitoring, Web-based Dashboard, Frida Framework, Microservices Architecture, Security Research, Behavioral Analysis, Process Monitoring